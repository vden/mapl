\chapter{Введение}
\label{chap:intro}

\epigraph{
1. Запишите задачу на бумаге.\\
2. Подумайте действительно хорошенько.\\
3. Запишите решение задачи.
}{<<Алгоритм Фейнмана>>, по Мюррею Гелл-Манну.}

\begin{lstlisting}[caption={An Iterative Solution to the Longest Common Subsequence (LCS)}]
def lcs(a,b):
    n, m = len(a), len(b)
    pre, cur = [0]*(n+1), [0]*(n+1) # Previous/current row
    for j in range(1,m+1): # Iterate over b
        pre, cur = cur, pre # Keep prev., overwrite cur.
        for i in range(1,n+1): # Iterate over a
            if a[i-1] == b[j-1]: # Last elts. of pref. equal?
                cur[i] = pre[i-1] + 1 # L(i,j) = L(i-1,j-1) + 1
            else: # Otherwise...
                cur[i] = max(pre[i], cur[i-1]) # max(L(i,j-1),L(i-1,j))
    return cur[n] # L(n,m)
\end{lstlisting}


%%% Local Variables: 
%%% mode: latex
%%% TeX-master: "mapl"
%%% End: 
