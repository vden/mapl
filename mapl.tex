%\RequirePackage{etex}
\documentclass{book}

\usepackage[utf8]{inputenc}
\usepackage[russian]{babel}
\usepackage{epigraph}
\usepackage{fancyvrb}
\usepackage{amsfonts}
\usepackage{amsmath}
\usepackage{amssymb}
\usepackage{url}
\usepackage[dvips]{color}
\usepackage[arrow,curve,matrix,frame]{xy}
\usepackage{indentfirst}	% Русская типографика требует красную строку
%\usepackage{MnSymbol}
\usepackage{array}
\usepackage{ifmtarg}
\usepackage{centernot}
\usepackage{empheq}
\usepackage{needspace}
\usepackage{colonequals}
\usepackage{graphicx}
\usepackage{tikz}
\usetikzlibrary{arrows,positioning,shapes,shadows,trees}
%\usepackage{tikz-qtree}
\usepackage{mdwlist}
\usepackage{Khartiya}

%\usepackage{fancybox}
%% Я хотел использовать fancybox ради \ovalbox, но он определяет свой
%% собственный Verbatim, который конфликтует с fancyvrb.
%% В конце концов нужно будет либо выдернуть оттуда определение
%% \ovalbox, либо отказаться от fancyvrb. Пока что я просто не делаю
%% круглых углов.
\newcommand{\ovalbox}[1]{\fbox{#1}}

\setlength{\epigraphwidth}{7cm}

\usepackage{longtable,citehack,enumerate}
\usepackage{tocbibind}
%\usepackage[unicode,colorlinks=true,a4paper]{hyperref}
\usepackage[unicode,a4paper]{hyperref}
\definecolor{pfpcolorint}{rgb}{0,0.4,0}
\definecolor{pfpcolorext}{rgb}{0,0,0.6}

\hypersetup{colorlinks,
	linkcolor=pfpcolorint,
	citecolor=pfpcolorint,
	filecolor=pfpcolorext,
	urlcolor=pfpcolorext,
	bookmarksopen,bookmarksopenlevel=0,
	pdffitwindow,
	pdftitle={Магнус Лай Хетланд. Алгоритмы в Python},
	% pdfsubject={},
	% pdfkeywords={}
}

\interfootnotelinepenalty=10000
% 70х100/16 -> 170mmх240mm
\usepackage[papersize={165mm,240mm},hmargin={1.6cm,2cm},vmargin={2cm,2cm},twoside]{geometry}

\usepackage[dvipdfm]{accsupp} % Контролирует эффект копирования из PDF

% headers & footers
\usepackage{fancyhdr}
\pagestyle{fancy}
%\include{version}
\fancyhead{}
\fancyhead[LE,RO]{\slshape{\rightmark{}}}
%\fancyhead[LE,RO]{\thepage{}}
%\fancyfoot{}
%\fancyfoot[LE,RO]{rev. \svnInfoMaxRevision}
% 
\usepackage{graphicx}

\usepackage[round]{natbib}
\bibliographystyle{plainnat}
\renewcommand{\cite}{\citep*}
\newcommand{\citeterse}{\citep}
\newcommand{\citeyr}[1]{(\citeyear{#1})}
\newcommand{\citeyropen}[1]{\citeyear{#1}}
\newcommand{\fulllongcite}[1]{\citeauthor*{#1} \citeyearpar{#1}}
\newcommand{\citeopen}[1]{\citealp*{#1}}
\newcommand{\citeopenterse}[1]{\citealp{#1}}

%%\usepackage{algorithm}

\usepackage{listings}
\lstloadlanguages{Python,bash}
\definecolor{listingcomment}{gray}{0.3}
\lstset{%
  basicstyle=\ttfamily,
  showtabs=false, 
  showspaces=false, 
  showstringspaces=false,
  tabsize=2,
  language=Python,
  commentstyle=\color{listingcomment},
  extendedchars=\true{},
  mathescape=\true{},% allows to use math symbols in listings - may be
                     % dangerous
  keywordstyle={}}
% замена буквосочетаний на симпатичные символы
\lstset{literate={->}{{$\to{}$}}1 {=>}{{$\Rightarrow$}}2 {|>}{{$\triangleright{}$}}1}

% \lstinline мешается в таблицах (пока не понял, почему) -- goga
%\newcommand{\code}[1]{\lstinline[mathescape]!#1!}
\newcommand{\code}[1]{\texttt{#1}}

%% proper captions for figures, etc
\RequirePackage{caption}
\captionsetup{labelsep=period,labelfont=bf}

% Добавляем точку после номера раздела
\usepackage{titlesec}
\titlelabel{\thetitle.\quad}
\usepackage[dotinlabels]{titletoc}


% settings for floating objects
\renewcommand{\topfraction}{.9}
\renewcommand{\textfraction}{.1}
\renewcommand{\bottomfraction}{.9}
%\renewcommand{\floatpagefraction}{.9}
%\renewcommand{\dblfloatpagefraction}{\floatpagefraction}
% \setcounter{topnumber}{4}
% \setcounter{bottomnumber}{4}
% \setcounter{totalnumber}{4}
% \setlength{\floatsep}{8pt plus 2pt minus 2pt}
% \setlength{\textfloatsep}{8pt plus 2pt minus 2pt}
% \setlength{\intextsep}{12pt plus 2pt minus 2pt}

% Custom commands
\newcommand{\<}{\langle}
\renewcommand{\>}{\rangle}
\newcommand{\nbdash}{\mbox{-}}
\newcommand{\nosol}{\nrightarrow}
\newcommand{\recommended}{\textsc{Рекомендуется}}

\newcommand{\lambdaterm}[1]{\mbox{\code{#1}}}
\newcommand{\nbcode}[1]{\mbox{\code{#1}}}
\newcommand{\ocode}[1]{\overline{\vphantom{A}\code{#1}}}

\newcommand{\defterm}[1]{\emph{#1}}

\newcommand{\ruleid}[1]{\mbox{\textsc{#1}}}
\newcommand{\ruletag}[1]{\tag{\textsc{#1}}\index{правило!\textsc{#1}}}

\newcommand{\N}{\mathbb{N}}
\newcommand{\dom}{\textit{dom}}
\newcommand{\range}{\textit{range}}
\newcommand{\fail}{\textit{неудача}}
\newcommand{\function}[1]{\mbox{\textit{#1}\,}}
\newcommand{\interpreter}[1]{\texttt{#1}\index{интерпретатор!\texttt{#1}}}

\newcommand{\Fsub}{F$_{\code{<:}}$}
\newcommand{\Fomega}{F$_\omega$}
\newcommand{\Fsubomega}{F$_{\code{<:}}^\omega$}

\renewcommand{\L}{\ensuremath{\lambda}}

\newcommand{\eqdef}{\stackrel{\text{\tiny def}}{=}}

\newcommand{\size}[1]{\bar #1 \bar}

\newcommand{\varrow}{\mbox{\strut\ensuremath{\:\Mapstochar\shortrightarrow\:}}}
\newcommand{\notsub}{\ensuremath{\centernot{\code{\textnormal{<:}}}}}

\definecolor{gray}{gray}{0.75}
\newcommand{\graybg}[1]{\colorbox{gray}{\mbox{#1}}}

\makeatletter
\newcommand{\translationnotemark}{{\renewcommand{\@makefnmark}{\mbox{$^*$}}\footnotemark{}}}
\newcommand{\translationnotetext}[1]{{\renewcommand{\@makefnmark}{\mbox{$^*$}}\footnotetext{#1~---~{\it прим. перев.}}}\addtocounter{footnote}{-1}}
\newcommand{\translationnote}[1]{\translationnotemark{}\translationnotetext{#1}}
\newcommand{\unmarkednote}[1]{{\renewcommand{\@makefnmark}{}\footnotemark{}\footnotetext{#1}}\addtocounter{footnote}{-1}}
\makeatother

\newcommand{\wearelazy}{\translationnote{При верстке перевода это правило не соблюдалось.}}

\newtheorem{definition}{\texttt{Определение}}[section]
\newtheorem{exercise}[definition]{\texttt{Упражнение}}
\newtheorem{axiom}[definition]{\texttt{Аксиома}}
\newtheorem{proposition}[definition]{\texttt{Утверждение}}
\newtheorem{lemma}[definition]{\texttt{Лемма}}
\newtheorem{theorem}[definition]{\texttt{Теорема}}
\newtheorem{convention}[definition]{\texttt{Соглашение}}
\newtheorem{corollary}[definition]{\texttt{Следствие}}
\newtheorem{example}[definition]{\texttt{Пример}}
\newtheorem{examples}[definition]{\texttt{Примеры}}

\newcommand{\errata}[1]{\relax}

\newcommand{\muheight}{\ensuremath{\mu\mbox{-height}}}
\newcommand{\recordc}[4]{\ensuremath{\code{#1}_#2{}^{#2\in #3..#4}}}

\newenvironment{exsolution}[1]
  {\par\smallskip\noindent
   \ref{#1}. \textsc{Решение:}}{}

%%%
%%% НОВАЯ ВЕРСТКА ТАБЛИЦ
%%%
%%% сначала идёт параграф с заголовком...
\newcommand{\displayNew}[3]{#1

%%% ... потом верхняя рамка
\rule[-1ex]{.4pt}{1ex}\rule{\mapltablewidth}{.4pt}\rule[-1ex]{.4pt}{1ex}\smallskip

%%% ... потом две minipage, стоящие рядом через черточку
\begin{minipage}[t]{.5\mapltablewidth}
#2
\end{minipage}
\vline
\begin{minipage}[t]{.5\mapltablewidth}
#3
\end{minipage}

%%% ... потом нижняя рамка
\rule{.4pt}{1ex}\rule{\mapltablewidth}{.4pt}\rule{.4pt}{1ex}\par
}

\makeatletter
\newcommand{\tablesection}[2]{\textit{#1}\@ifmtarg{#2}{\relax}{\hfill\fbox{#2}}}
\makeatother

\newcommand{\displayOneCol}[2]{
  { % hold \arraystretch redefinition inside the display
    #1 \\
    \renewcommand{\arraystretch}{0}
    \begin{tabular}{|c|}
      \hline
      \rule{0pt}{1ex} \\
      \multicolumn{1}{c}{
        \renewcommand{\arraystretch}{1}
        \begin{tabular}{l}
          #2
        \end{tabular}
      } \\
      \rule{0pt}{1ex} \\
      \hline
    \end{tabular}
  }
}



\title{Алгоритмы в Python. Проектирование алгоритмов на языке Python}
\author{Магнус Лай Хетланд}

\begin{document}
\newlength{\mapltablewidth}
\setlength{\mapltablewidth}{\linewidth}
\addtolength\mapltablewidth{-2.8pt}
\VerbatimFootnotes
\selectlanguage{russian}%
\frenchspacing
\righthyphenmin=2
\sloppy
\setcounter{topnumber}{1}

\maketitle

%%% здесь можно комментировать одно и раскомментировать другое, чтобы
%%% быстро тестировать компиляцию по главам
%\input{chapter-13.tex}
\tableofcontents

\chapter*{Об авторе\markright{Об авторе}{}}
\addcontentsline{toc}{chapter}{Об авторе}

Магнус Лай Хетланд "--- опытный программист на Python, использующий этот язык с 90-х годов. Также он читает курс теории алгоритмов в Норвежском университете науки и технологий и изучает алгоритмы почти десять лет. Хетланд "--- автор книги <<Beginning Python>>.

\chapter*{О техническом рецензенте\markright{О техническом рецензенте}{}}
\addcontentsline{toc}{chapter}{О техническом рецензенте}

Алекс Мартелли родился и вырос в Италии и имеет степень доктора электротехники Болонского университета. Он написал книгу <<Python in a nutshell>> и был соавтором книги <<Python Cookbook>>. Является членом PSF, в 2002 получил премию Activators’ Choice Award, а в 2006 "--- премию Frank Willison Award за вклад в развитие сообщества Python. Сейчас живет в Калифорнии и работает ведущим разработчиком в Google. Больше о нем можно узнать по адресу: www.google.com/profiles/aleaxit; биографию можно найти на http://en.wikipedia.org/wiki/Alex\_Martelli.


\chapter*{Благодарности\markright{Благодарности}{}}
\addcontentsline{toc}{chapter}{Благодарности}

Спасибо всем, кто так или иначе внес свой вклад в написание этой книги. Конечно же, это мои наставники по теории алгоритмов, Арне Халаас и Бьорн Олстад, вся команда Apress и мой великолепный технический рецензент Алекс. Спасибо Нильсу Гримсмо, Йону Мариусу Венстаду, Оле Эдсбергу, Рольву Сеехуусу и Йоргу Родсо за полезные замечания. Спасибо моим родителям, Керсти Лай и Тору М. Хетланду, и моей сестре Анне Лай-Хетланд за их интерес и поддержку, а также моему дяде Акселю за проверку моего французского. И в заключение огромная благодарность Python Software Foundation за разрешение использовать части стандартной библиотеки Python и Рэндаллу Манро за разрешение включить в книгу несколько его чудесных комиксов XKCD.


\chapter*{Предисловие\markright{Предисловие}{}}
\addcontentsline{toc}{chapter}{Предисловие}

Эта книга родилась как слияние трех моих увлечений: алгоритмов, программирования на Python и объяснения людям разных вещей. По мне эстетика подразумевает три действия: вы находите правильный способ сделать что-либо, приглядываетесь к нему до тех пор, пока не проявится намек на элегантность, а затем шлифуете его до блеска. Ну, или хотя бы делаете чуть красивее. Конечно, когда требуется выполнить очень много работы, вы, вероятно, не сможете довести свои решения до совершенства. К счастью, большая часть материалов в этой книге близка к совершенству, так как я описываю действительно прекрасные алгоритмы и доказательства и использую один из лучших языков программирования. Что же до третьей составляющей, я серьезно поработал, чтобы объяснить вещи так просто, как это только возможно. Но даже при этом, я уверен, что во многих местах этого не достиг, так что если у вас есть советы по улучшению книги, я буду рад их услышать. Кто знает, может быть какие-то из ваших идей могут войти в следующее издание? Сейчас же я надеюсь, что эта книга вам понравится и вдохновит вас двигаться дальше. Если вы можете, используйте ее, чтобы сделать мир чуточку лучше.
%%% Local Variables: 
%%% mode: latex
%%% TeX-master: "mapl"
%%% End: 


\chapter{Введение}
\label{chap:intro}

\epigraph{
1. Запишите задачу на бумаге.\\
2. Подумайте действительно хорошенько.\\
3. Запишите решение задачи.
}{<<Алгоритм Фейнмана>>, по Мюррею Гелл-Манну.}


Представьте себе такую задачу. Вам нужно посетить все города, городки и поселки, например, Швеции и вернуться в исходную точку. Это может занять немало времени (ведь нужно заглянуть в 27 978 мест), так что Вам хочется сделать маршрут покороче. Вы собираетесь заехать в каждое место только один раз следуя кратчайшим путем. Будучи программистом Вы точно не собираетесь рисовать маршрут вручную. Вместо этого Вы попробуете написать программу, которая все рассчитает сама. Однако, по какой-то причине Вам это никак не удается. Очевидное решение неплохо работает на небольшом количестве городов, но на реальной задаче расчет никак не завершается, а улучшить программу оказывается неожиданно сложно. Как быть?

Между тем, в 2004-ом году команда из 5 исследователей\footnote{David Applegate, Robert Bixby, Vašek Chvátal, William Cook, and Keld Helsgaun} нашла такой путь по Швеции, после того как несколько подобных проектов завершились неудачей. Эта команда использовала самое новое ПО с кучей хитрых оптимизаций, запущенное на кластере из 96 машин на Xeon 2.6ГГц. Их программа проработала с марта 2003-го до мая 2004-го, прежде чем было получено оптимальное решение. Ребята посчитали, что с учетом разных перерывов общее процессорное время, потраченное на решение задачи, оказалось равным \textit{85 годам}!

Представьте теперь похожую задачу: из Кашгара, на самого западе Китая, Вам нужно попасть в Нинбо, на восточное побережье, следуя кратчайшем путем. В Китае 3 583 715 км автомобильных и 77 384 км железных дорог, которые пересекаются в миллионах мест и образуют неисчислимое количество возможных маршрутов. Может показаться, что эта проблема аналогична предыдущей, но уже сейчас такая задача \textit{поиска кратчайшего пути} решается без каких-либо затруднений и задержек GPS-навигаторами и онлайн-картами. Если Вы отметите два этих города на любимом картографическом сервисе, то получите кратчайший маршрут за пару мгновений. Так в чем же тут дело?

Дальше в книге Вы узнаете больше о двух этих задачах. Первая называется \textit {проблемой коммивояжера} и описана в главе \ref{chap:hard-problems}, а так называемая проблема поиска кратчайшего пути рассматривается в главе \ref{chap:from-a-to-b}. Я надеюсь, что Вы начнете хорошо понимать, почему одни задачи так сложны для решения, в то время как другие имеют несколько известных и эффективных решений. Что еще важнее, Вы научитесь работать с любыми вычислительными и алгоритмическими задачами, либо эффективно решая их, используя описанные в книге техники и алгоритмы, либо показывая, что они слишком сложны, и приближенное решение "--- это лучшее, на что можно рассчитывать. Эта глава вкратце описывает всю книгу: что Вы можете от нее получить и что требуется от Вас. В ней также обозначены темы остальных глав, чтобы Вы могли читать их выборочно.


\begin{lstlisting}[caption={An Iterative Solution to the Longest Common Subsequence (LCS)}]
def lcs(a,b):
    n, m = len(a), len(b)
    pre, cur = [0]*(n+1), [0]*(n+1) # Previous/current row
    for j in range(1,m+1): # Iterate over b
        pre, cur = cur, pre # Keep prev., overwrite cur.
        for i in range(1,n+1): # Iterate over a
            if a[i-1] == b[j-1]: # Last elts. of pref. equal?
                cur[i] = pre[i-1] + 1 # L(i,j) = L(i-1,j-1) + 1
            else: # Otherwise...
                cur[i] = max(pre[i], cur[i-1]) # max(L(i,j-1),L(i-1,j))
    return cur[n] # L(n,m)
\end{lstlisting}


%%% Local Variables: 
%%% mode: latex
%%% TeX-master: "mapl"
%%% End: 
 %% Introduction
\chapter{Основы}
\label{chap:basics}

\epigraph{Трейси: Я не знал, что ты была там.\\
Зоуи: В каком-то смысле. Невидимость — ты, вероятно, слышал о ней.\\
Трейси: Не думаю, что это было в базовом курсе.
}{Из 14-го эпизода <<Firefly>>, <<Сообщение>>}



\section{Некоторые базовые понятия в вычислениях}

\section{Асимптотическая нотация}
\subsection{Это какая-то китайская грамота!}
\subsection{Основные термины и правила}
\subsection{Take the Asymptotics for a Spin }
\subsection{Three Important Cases}
\subsection{Эмпирическая оценка алгоритмов}
\label{sec:empirical-evaluation}

В этой книге описывается\textit{ проектирование алгоритмов} (и тесно связанный с ним\textit{ анализ алгоритмов}). Но в разработке есть также и другой немаловажный процесс, жизненно необходимый при создании крупных реальных проектов, это — \textit{оптимизация алгоритмов}, искусство их \textit{эффективной реализации}. С точки зрения такого разделения, проектирование алгоритма можно рассматривать как способ достижения низкой асимптотической сложности алгоритма (с помощью разработки эффективного алгоритма), а оптимизацию — как уменьшение констант, скрытых в этой асимптотике.

Хотя я могу дать несколько советов по алгоритмам проектирования в Python, сложно угадать, какие именно хитрости и уловки дадут вам лучшую производительность в конкретной задаче, над которой вы работаете, или для вашего оборудования и версии Python. (Асимптотики используются как раз для того, чтобы не было нужды прибегать к таким вещам). Вообще, в некоторых случаях хитрости могут вовсе не потребоваться, потому что ваша программа и так достаточно быстра. Самое простое, что вы можете сделать в большинстве случаев, это просто пробовать и смотреть. Если у вас \textit{есть идея} какого-то хака, то просто опробуйте ее! Реализуйте свою хитрость и запустите несколько тестов. Есть ли улучшения? А если это изменение делает ваш код менее читаемым, а прирост производительности мал, то подумайте — стоит ли оно того?

\begin{note}
Этот раздел рассказывает об оценке алгоритмов, а не об их оптимизации. В приложении \ref{app:A} есть несколько советов по ускорению программ на Python.
\end{note}

Есть ряд теоретических аспектов так называемой экспериментальной алгоритмики (то есть, экспериментальной оценки алгоритмов и их реализации), но они выходят за рамки этой книги, так что я дам вам несколько практических советов, которые могут быть весьма полезны.

\subsubsection*{Совет 1. Если возможно, то не беспокойтесь об этом}

Беспокойство об асимптотической сложности может быть очень полезным. Иногда \textit{решение} задачи из-за сложности на практике может \textit{перестать} быть таковым. Однако, постоянные константы в асимптотике часто совсем не критичны. Попробуйте простую реализацию алгоритма для начала, и убедитесь, что она работает стабильно. (В принципе, вы можете сначала попробовать примитивный алгоритм. Гуру программирования Кен Томпсон писал: «Когда вы в затруднении, используйте перебор вариантов». Перебор вариантов в алгоритмах обычно означает попытку попробовать каждое из возможных решений, при этом временные затраты будут безумными!) Если все работает — оставьте как есть.

\subsubsection*{Совет 2. Для измерения времени работы используйте \texttt{timeit}}

Модуль \texttt{timeit} предназначен для измерения времени работы. Хотя для получения действительно надежных данных вам потребуется выполнить кучу работы, для практических целей timeit вполне сгодится. Например:

\begin{lstlisting}
>>> import timeit
>>> timeit.timeit("x = 2 + 2")
0.034976959228515625
>>> timeit.timeit("x = sum(range(10))")
0.92387008666992188
\end{lstlisting}

Модуль timeit можно использовать прямо из командной строки, например:

\begin{lstlisting}
python -m timeit -s"import mymodule as m" "m.myfunction()"
\end{lstlisting}

Существует кое-что, с чем вы должны быть осторожны при использовании \texttt{timeit}: побочные эффекты, которые будут влиять на повторное исполнение \texttt{timeit}. Функция \texttt{timeit} будет запускать ваш код несколько раз, чтобы увеличить точность, и если прошлые запуски влияют на последующие, то вы окажетесь в затруднительном положении. Например, если вы измеряете скорость выполнения чего-то вроде \texttt{mylist.sort()}, список будет отсортирован только \textit{в первый раз}. Во время остальных тысяч запусков список уже будет отсортированным и это даст нереально маленький результат.

Больше информации об этом модуле и о том, как он работает, можно найти в документации стандартной библиотеки Python.

\subsubsection*{Совет 3. Чтобы найти узкие места, используйте профайлер}

Часто, для того чтобы понять, какие части программы требуют оптимизации, делаются разные догадки и предположения. Такие предположения нередко оказываются ошибочными. Вместо того, чтобы гадать, воспользуйтесь профайлером. В стандартной поставке Python идет несколько профайлеров, но рекомендуется использовать \texttt{cProfile}. Им так же легко пользоваться, как \texttt{timeit}, но он дает больше подробной информации о том, на что тратится время при выполнении программы. Если основная функция вашей программы называется \texttt{main}, вы можете использовать профайлер следующим образом:

\begin{lstlisting}
import cProfile
cProfile.run('main()')
\end{lstlisting}

Такой код выведет отчет о времени работы различных функций программы. Если в вашей системе нет модуля \texttt{cProfile}, используйте \texttt{profile} вместо него. Больше информации об этих модулях можно найти в документации. Если же вам не так интересны детали \textit{реализации}, а просто нужно оценить поведение \textit{алгоритма} на конкретном наборе данных, воспользуйтесь модулем \texttt{trace} из стандартной библиотеки. С его помощью можно посчитать, сколько раз будут выполнены то или иное выражение или вызов в программе.

\subsubsection*{Совет 4. Показывайте результаты графически}

Визуализация может стать прекрасным способом, чтобы понять что к чему. Для исследования производительности часто применяются \textit{графики} (например, для оценки связи количества данных и времени исполнения) и диаграммы типа \textit{<<ящик с усами>>}, отображающие распределение временных затрат при нескольких запусках. Примеры можно увидеть на рисунке \ref{fig:evaluation-diagrams}.

\begin{figure}[h]
	\centering
	\includegraphics[width=\textwidth]{img/2-2.png}
	\caption{Визуализация времени работы для программ A, B и C и размеров данных 10 "--- 50}
	\label{fig:evaluation-diagrams}
\end{figure}

Отличная библиотека для построения графиков и диаграмм из Python "--- matplotlib (можно взять на http://matplotlib.sf.net).

\subsubsection*{Совет 5. Будьте внимательны в выводах, основанных на сравнении времени работы}

Этот совет довольно расплывчатый, потому что существует много ловушек, в которые можно попасть, делая выводы о том, какой способ лучше, на основании сравнения времени работы. Во-первых, любая разница, которую вы видите, может определяться случайностью. Если вы используете специальные инструменты вроде \texttt{timeit}, риск такой ситуации меньше, потому что они повторяют измерение времени вычисления выражения несколько раз (или даже повторяют весь замер несколько раз, выбирая лучший результат). Таким образом, всегда будут случайные погрешности и если разница между двумя реализациями не превышает некоторой погрешности, нельзя сказать, что эти реализации различаются (хотя и то, что они одинаковы, тоже \textit{нельзя} утверждать).

\begin{note}
Если вам нужно сделать выводы быстро, можно воспользоваться методом проверки статистических гипотез. Однако, на практике, если различия действительно очень малы, то, вероятно, неважно, какую реализацию вы выберете, так что используйте ту, что больше нравится.
\end{note}

Проблема усложняется, если вы сравниваете больше двух реализаций. Количество пар для сравнения увеличивается пропорционально \textit{квадрату} количества сравниваемых версий, \textit{сильно} увеличивая вероятность того, что как минимум две из версий будут казаться слегка различными. (Это называется проблемой \textit{множественного сравнения}). Существуют статистические решения этой проблемы, но самый простой способ "--- повторить эксперимент с двумя подозрительными версиями. Возможно, потребуется сделать это несколько раз. Они по-прежнему выглядят похожими?

Во-вторых, есть несколько моментов, на которые нужно обращать внимание при сравнении средних величин. Как минимум, вы должны сравнивать средние значения реального времени работы. Обычно, чтобы получить показательные числа при измерении производительности, время работы каждой программы нормируется делением на время выполнения какого-нибудь стандартного, простого алгоритма. Действительно, это может быть полезным, но в ряде случаев сделает бессмысленными результаты замеров. Несколько полезных указаний на эту тему можно найти в статье <<How not to lie with statistics: The correct way to summarize benchmark results>> Fleming и Wallace. Также можно почитать Bast и Weber <<Don't compare averages>>, или более новую статью Citron и др. <<The harmonic or geometric mean: does it really matter?>>

И в-третьих, возможно, ваши выводы нельзя обобщать. Подобные измерения на другом наборе входных данных и на другом железе могут дать другие результаты. Если кто-то будет пользоваться результатами ваших измерений, \textit{необходимо последовательно задокументировать}, каким образом вы их получили.

\subsubsection*{Совет 6. Будьте осторожны, делая выводы об асимптотике из экспериментов}

Если вы хотите что-то сказать окончательно об асимптотике алгоритма, то необходимо проанализировать ее, как описано ранее в этой главе. Эксперименты могут дать вам намеки, но они очевидно проводятся на конечных наборах данных, а асимптотика "--- это то, что происходит при сколь угодно больших размерах данных. С другой стороны, если только вы не работаете в академической сфере, \textit{цель} асимптотического анализа "--- сделать какой-то вывод о поведении алгоритма, реализованного конкретным способом и запущенного на определенном наборе данных, а это значит, что измерения \textit{должны быть} соответствующими.

Например, вы \textit{предполагаете}, что алгоритм работает с квадратичной сложностью, но вы не можете окончательно доказать это. Можете ли вы использовать эксперименты для доказательства вашего предположения? Как уже говорилось, эксперименты (и оптимизация алгоритмов) имеют дело в основном с постоянными коэффициентами, но \textit{выход есть}. Основной проблемой является то, что ваша гипотеза на самом деле непроверяема экспериментально. Если вы утверждаете, что алгоритм имеет сложность $O(n^2)$, то данные не могут это ни подтвердить, ни опровергнуть. Тем не менее, если вы сделаете вашу гипотезу более \textit{конкретной}, то она станет проверяемой. Вы могли бы, например, основываясь на некоторых данных положить, что время работы программы никогда не будет превышать $0.24n^2+0.1n+0.03$ секунд в вашем окружении. Это проверяемая (точнее, \textit{опровергаемая}) гипотеза. Если вы сделали множество измерений, но так и не можете найти контр-примеры, значит ваша гипотеза может быть верна. А это уже и подтверждает гипотезу о квадратичной сложности алгоритма.

\section{Реализация графов и деревьев }
\label{sec:implementing-graphs-and-trees}
Первая задача из главы \ref{chap:intro}, в которой нам требовалось объехать Швецию и Китай, была примером задачи, которая может быть решена с помощью одного из мощнейших инструментов "--- с помощью \textit{графов}. Часто, если вы можете определить, что решаете задачу на графы, вы по-крайней мере на полпути к решению. А если ваши данные можно каким-либо образом представить как \textit{деревья}, у вас есть все шансы построить действительно \textit{эффективное} решение.

Графами можно представить любую структуру или систему, от транспортной сети до сети передачи данных и от взаимодействия белков в ядре клетке до связей между людьми в Интернете.
Ваши графы могут стать еще полезнее, если вы добавите в них дополнительные данные вроде \textit{весов} или \textit{расстояний}, что даст возможность описывать такие разнообразные проблемы как игру в шахматы или определение подходящей работы для человека в соответствии с его способностями.
Деревья — это просто особый вид графов, так что большинство алгоритмов и представлений графов сработают и для них.
Однако, из-за их особых свойств (связанность и отсутствие циклов), можно применить специальные (и весьма простые) версии алгоритмов и представлений.
На практике в некоторых случаях встречаются структуры (такие как XML-документы или иерархия каталогов), которые могут быть представлены в виде деревьев\footnote{С учетом атрибутов IDREF и символьных ссылок XML-документы и иерархия каталогов становятся собственно графами.}. На самом деле эти <<некоторые>> случаи довольно-таки общие.

Если вы забыли терминологию (или ее и не знали), почитайте приложение \ref{app:graph-theory}, <<Терминология теории графов>>. Здесь же опишем основные моменты:

\begin{itemize}
\item Граф $G = (V, E)$ состоит из \textit{вершин}, $V$, и \textit{ребер} между ними, $E$. Если ребра имеют направление, то граф называется \textit{направленным}.
\item Вершины, связанные ребром, называются \textit{смежными}. Соединяющее две вершины ребро называется \textit{инцидентным} с ними. Вершины, смежные с $v$ называются \textit{соседними} с $v$.
\item \textit{Подграф} графа $G = (V,E)$ состоит из подмножества $V$ и подмножества $E$. \textit{Путь} в графе "--- это подграф, вершины которого соединены ребрами последовательно, причем каждая вершина включена только один раз. \textit{Цикл} — это то же самое, что и путь, только последнее его ребро связывает последнюю вершину с первой.
\item Если каждому ребру в $G$ будет сопоставлено определенное значение (\textit{вес}), то $G$ будет называться \textit{взвешенным} графом. \textit{Длина} пути или цикла — это сумма всех весов его ребер или, для невзвешенных графов, просто количество ребер.
\item \textit{Лесом} называется граф без циклов, а связанный граф — это \textit{дерево}. Иными словами, лес состоит из одного или многих деревьев.
\end{itemize}

Описание задачи в терминах графов является довольно абстрактным, так что если вам нужно реализовать решение, вы должны представить графы в виде каких-либо структур данных.

\subsection{Adjacency Lists and the Like}
\subsection{Матрицы смежности}
\label{sec:adjacency-matrix}

Другая распространенная форма представления графов — это матрицы смежности. Основное отличие их в следующем: вместо перечисления всех смежных с каждой из вершин, мы записываем один ряд значений (массив), каждое из которых соответствует возможной смежной вершине (есть хотя бы одна такая для каждой вершины графа), и сохраняем значение (в виде \texttt{True} или \texttt{False}), показывающее, действительно ли вершина является смежной. И вновь простейшую реализацию можно получить используя вложенные списки, как видно из листинга \ref{lst:adjacency}. Заметьте, что это также требует, чтобы вершины были пронумерованы от $0$ до $V-1$. В качестве значений истинности используются 1 и 0 (вместо \texttt{True} и \texttt{False}), чтобы сделать матрицу читабельной.

\begin{lstlisting}[caption={Матрица смежности, реализованная с помощью вложенных списков},label={lst:adjacency}]
a, b, c, d, e, f, g, h = range(8)
	    # a b c d e f g h

N = [[0,1,1,1,1,1,0,0], # a
		 [0,0,1,0,1,0,0,0], # b
		 [0,0,0,1,0,0,0,0], # c
		 [0,0,0,0,1,0,0,0], # d
		 [0,0,0,0,0,1,0,0], # e
		 [0,0,1,0,0,0,1,1], # f
		 [0,0,0,0,0,1,0,1], # g
		 [0,0,0,0,0,1,1,0]] # h
\end{lstlisting}

Способ использования матриц смежности слегка отличается от списков и множеств смежности. Вместо проверки, входит ли $b$ в $N[a]$, вы будете проверять, истинно ли значение ячейки матрицы $N[a][b]$. Кроме того, больше нельзя использовать $len(N[a])$, чтобы получить количество смежных вершин, потому что все ряды одной и той же длины. Вместо этого можно использовать функцию \texttt{sum}:
\begin{lstlisting}
>>> N[a][b]
1
>>> sum(N[f])
3
\end{lstlisting}

Матрицы смежности имеют ряд полезных свойств, о которых стоит знать. Во-первых, так как мы не рассматриваем графы с петлями (т.е. не работаем с псевдографами), все значения на диагонали — ложны. Также, ненаправленные графы обычно описываются парами ребер в обоих направлениях. Это значит, что матрица смежности для ненаправленного графа будет симметричной.

Расширение матрицы смежности для использования весов тривиально: вместо сохранения логических значений, сохраняйте значения весов. В случае с ребром $(u, v)$ $N[u][v]$ будет весом ребра $w(u,v)$ вместо \texttt{True}. Часто в практических целях несуществующим ребрам присваиваются бесконечные веса. (Это гарантирует, что они не будут включены, например, в кратчайшие пути, т. к. мы ищем путь по существующим ребрам). Не всегда очевидно, как представить бесконечность, но совершенно точно есть несколько разных вариантов.

Один из них состоит в том, чтобы использовать некорректное для веса значение, такое как \texttt{None} или $-1$, если известно, что все веса неотрицательны. Возможно, в ряде случаев полезно использовать действительно большие числа. Для целых весов можно применить \texttt{sys.maxint}, хотя это значение и не обязательно самое большое (длинные целые могут быть больше). Есть и значение, введенное для отражения бесконечности: \texttt{inf}. Оно недоступно в Python напрямую по имени и выражается как \texttt{float('inf')}\footnote{Гарантируется, что это работает для Python 2.6 и старше. В ранних версиях подобные специальные значения были платформо-зависимы, хотя \texttt{float('inf')} или \texttt{float('Inf')} должны сработать на большинстве платформ.}.

Листинг \ref{lst:adjacency-weighted} показывает, как выглядит матрица весов, реализованная вложенными списками. Использованы те же веса, что и в листинге \ref{lst:adjlist-weighted}.

\begin{lstlisting}[caption={Матрица весов с бесконечными значениями для отсутствующих ребер}, label={lst:adjacency-weighted}]
a, b, c, d, e, f, g, h = range(8)
_ = float('inf')

		# a b c d e f g h

W = [[0,2,1,3,9,4,_,_], # a
	 	 [_,0,4,_,3,_,_,_], # b
		 [_,_,0,8,_,_,_,_], # c
		 [_,_,_,0,7,_,_,_], # d
		 [_,_,_,_,0,5,_,_], # e
		 [_,_,2,_,_,0,2,2], # f
		 [_,_,_,_,_,1,0,6], # g
		 [_,_,_,_,_,9,8,0]] # h
\end{lstlisting}

Бесконечное значение обозначено как подчеркивание (\texttt{\_}), потому что это коротко и визуально различимо. Естественно, можно использовать любое имя, которое вы предпочтете. Обратите внимание, что значения на диагонали по-прежнему равны нулю, потому что даже без учета петель, веса часто интерпретируются как расстояния, а расстояние от вершины до самой себя равно нулю.

Конечно, матрицы весов дают возможность очень просто получить веса ребер, но, к примеру, проверка смежности и определение степени вершины, или обход всех смежных вершин делаются иначе. Здесь нужно использовать бесконечное значение, примерно так (для большей наглядности определим \texttt{inf = float('inf')}):
\begin{lstlisting}
>>> W[a][b] < inf
True
# смежность
>>> W[c][e] < inf
False
# смежность
>>> sum(1 for w in W[a] if w < inf) - 1 # степень
5
\end{lstlisting}

Заметьте, что из полученной степени вычитается $1$, потому что мы не считаем значения на диагонали. Сложность вычисления степени тут $\Theta(n)$, в то время как в другом представлении и принадлежность, и степень вершины можно определить за константное время. Так что  вы всегда должны понимать, \textit{как именно} вы собираетесь использовать ваш граф и выбирать для него соответствующее представление.

\begin{notice}{Массивы специального назначения из NumPy}

Библиотека NumPy содержит много функциональности, связанной с многомерными массивами. Для представления графов большая ее часть не нужна, но массивы из NumPy весьма полезны, например, для реализации матриц весов или смежности.

Вместо создания пустой матрицы весов или смежности из списков для $n$ вершин, вроде такого:
\begin{lstlisting}
>>> N = [[0]*10 for i in range(10)]
\end{lstlisting}
в NumPy можно использовать функцию \texttt{zeros}:
\begin{lstlisting}
>>> import numpy as np
>>> N = np.zeros([10,10])
\end{lstlisting}

Отдельные элементы доступны по индексам, разделенным запятой: $A[u,v]$. Чтобы получить соседние с данной вершины, используется одиночный индекс: $A[u]$.

Пакет NumPy можно получить по адресу http://numpy.scipy.org.

Имейте ввиду, что вам нужна та версия NumPy, что будет работать с вашей версией Python. Если последний релиз NumPy не поддерживает ту версию Python, что вы используете, вы можете скомпилировать и установить библиотеку прямо из репозитория исходных кодов. Исходный код можно получить с помощью следующих команд (предполагается, что у вас установлена Subversion):

\texttt{svn co http://svn.scipy.org/svn/numpy/trunk numpy}

Больше информации о том, как компилировать и устанавливать NumPy, так же как и подробную документацию, можно найти на сайте библиотеки.

\end{notice}

\subsection{Реализация деревьев}

Любое представление графов, естественно, можно использовать для представления деревьев, потому что деревья "--- это особый вид графов. Однако, деревья играют свою большую роль в алгоритмах, и для них разработано много соответствующих структур и методов. Большинство алгоритмов на деревьях (например, поиск по деревьям, описанный в главе \ref{chap:divide-combine}) можно рассматривать в терминах теории графов, но специальные структуры данных делают их проще в реализации. 

Проще всего описать представление дерева с корнем, в котором ребра спускаются вниз от корня. Такие деревья часто отображают иерархическое ветвление данных, где корень отображает все объекты (которые, возможно, хранятся в листьях), а каждый внутренний узел показывает объекты, содержащиеся в дереве, корень которого "--- этот узел. Это описание можно использовать, представив каждое поддерево списком, содержащим все его поддеревья-потомки. Рассмотрим простое дерево, показанное на рисунке \ref{fig:simple-tree}.

Мы можем представить это дерево как список списков:
\begin{lstlisting}
>>> T = [["a", "b"], ["c"], ["d", ["e", "f"]]]
>>> T[0][1]
'b'
>>> T[2][1][0]
'e'
\end{lstlisting}

Каждый список в сущности является списком потомков каждого из внутренних узлов. Во втором примере мы обращаемся к третьему потомку корня, затем ко второму его потомку и в конце концов "--- к первому потомку предыдущего узла (этот путь отмечен на рисунке).


\tikzstyle{tree} = [circle,
						fill=white!90!black!10,
						drop shadow]
\begin{figure}[h]
\centering
\begin{tikzpicture}[level/.style={sibling distance=35mm/#1},style={>=latex,->}]
    \tikzstyle{every node}=[circle,draw,text centered,minimum height=2.5em]
    \node[tree,draw=red] (z) {}
        child { 
			node[tree] {}
			child { node[tree] {\Large $a$} }
			child {	node[tree] {\Large $b$} } 
		}
        child {
            node[tree] {}
			child { node[tree] {\Large $c$} }
        }
        child { 
			node[tree,draw=red] (1) {} 
			child { node[tree] {\Large $d$} }
			child { 
				node[tree,draw=red] (2) {}
				child { node[tree,draw=red] (3) {\Large $e$} }
				child { node[tree] {\Large $f$} }
			}
		}
    ;
	\draw[-latex,color=red] (z) .. controls +(east:0.5cm) and +(north west:1.5cm) .. (1) {};
	\draw[-latex,color=red] (1) .. controls +(south east:0.3cm) and +(north:1cm) .. (2) {};
	\draw[-latex,color=red] (2) .. controls +(south west:0.3cm) and +(north:1cm) .. (3) {};
\end{tikzpicture}
\caption{Пример дерева с отмеченным путем от корня к листу}
\label{fig:simple-tree}
\end{figure}


В ряде случаев возможно заранее определить максимальное число потомков каждого узла. (Например, каждый узел \textit{бинарного дерева} может иметь до двух потомков). Поэтому можно использовать другие представления, скажем, объекты с отдельным атрибутом для каждого из потомков как в листинге \ref{lst:binary-tree}.

\begin{figure}[h!]
\begin{lstlisting}[caption={Класс бинарного дерева и его использование}, label={lst:binary-tree}]
class Tree:
	def __init__(self, left, right):
		self.left = left
		self.right = right

>>> t = Tree(Tree("a", "b"), Tree("c", "d"))
>>> t.right.left
'c'
\end{lstlisting}
\end{figure}
Для обозначения отсутствующих потомков можно использовать \texttt{None} (в случае если у узла только один потомок). Само собой, можно комбинировать разные методы (например, использовать списки или множества потомков для каждого узла).

Распространенный способ реализации деревьев, особенно на языках, не имеющих встроенной поддержки списков, это так называемое представление <<первый потомок, следующий брат>>. В нем каждый узел имеет два <<указателя>> или атрибута, указывающих на другие узлы, как в бинарном дереве. Однако, первый из этих атрибутов ссылается на первого потомка узла, а второй "--- на его следующего брата\translationnote{т.~е. узел, имеющий того же родителя, но находящийся правее}. Иными словами, каждый узел дерева имеет указатель на связанный список его потомков, а каждый из этих потомков ссылается на свой собственный аналогичный список. Таким образом, небольшая модификация бинарного дерева (листинг \ref{lst:binary-tree}) даст нам многопутевое дерево, показанное в листинге \ref{lst:multiway-tree}.

\begin{figure}[h]
\begin{lstlisting}[caption={Класс многопутевого дерева}, label={lst:multiway-tree}]
class Tree:
	def __init__(self, kids, next=None):
		self.kids = self.val = kids
		self.next = next
\end{lstlisting}
\end{figure}

Отдельный атрибут \texttt{val} здесь введен просто для того, чтобы получить понятный вывод при указании значения (например, <<c>>) вместо ссылки на узел. Естественно, все это можно изменять. Вот пример того, как можно обращаться с этой структурой:
\begin{lstlisting}
>>> t = Tree(Tree("a", Tree("b", Tree("c", Tree("d")))))
>>> t.kids.next.next.val
'c'
\end{lstlisting}

А вот как выглядит это дерево:
\begin{figure}[!h]
\centering
\begin{tikzpicture}[sibling distance=20mm,style={>=latex,->}]
    \tikzstyle{every node}=[circle,draw,text centered,minimum height=2.5em]
    \node[tree] (z) {}
		child { node[tree] (a) {\Large $a$} }
		child {	node[tree] (b) {\Large $b$} } 
		child { node[tree] (c) {\Large $c$} }
		child { node[tree] (d) {\Large $d$} }
    ;
	\draw[-latex,dashed,color=white!20!black!80] (z) .. controls +(west:0.5cm) and +(north east:1.5cm) .. (a) {};
	\draw[-latex,dashed,color=white!20!black!80] (a) -- (b) {};
	\draw[-latex,dashed,color=white!20!black!80] (b) -- (c) {};
	\draw[-latex,dashed,color=white!20!black!80] (c) -- (d) {};
\end{tikzpicture}
\end{figure}

Атрибуты \texttt{kids} и \texttt{next} показаны пунктирными линиями, а сами ребра дерева "--- сплошными. Я немного схитрил и не стал показывать отдельные узлы для строк <<a>>, <<b>> и т.д. Вместо этого я трактую их как метки на соответствующих родительских узлах. В более сложном дереве вместо использования одного атрибута и для хранения значения узла и для ссылки на список потомков, для обеих целей могут понадобиться отдельный атрибуты. Обычно для обхода дерева используется более сложный код (включая циклы и рекурсию), чем приведенный здесь с жестко заданным путем. Больше об этом написано в главе \ref{chap:traversal}. В главе \ref{chap:combine-divide} также обсуждаются многопутевые деревья и балансировка деревьев.

\begin{notice}{Шаблон проектирования <<Набор>>}
При проектировании (да и реализации) таких структур данных как деревья может оказаться полезным гибкий класс, позволяющий задавать набор атрибутов через конструктор. Здесь нас может выручить шаблон проектирования <<Набор>> (названный так Алексом Мартелли в <<Python Cookbook>>). Есть много способов его реализации, но суть видна из следующего кода:
\begin{lstlisting}
class Bunch(dict):
	def __init__(self, *args, **kwds):
		super(Bunch, self).__init__(*args, **kwds)
		self.__dict__ = self
\end{lstlisting}

Есть несколько полезных способов его применения. Во-первых, он позволяет создать и задать значения атрибутов, передав их как аргументы при создании объекта:
\begin{lstlisting}
>>> x = Bunch(name="Jayne Cobb", position="PR")
>>> x.name
'Jayne Cobb'
\end{lstlisting}

Во-вторых, наследование от \texttt{dict} дает много дополнительного функционала, такого как получение всех ключей (атрибутов) или простая проверка наличия атрибута. Вот пример:
\begin{lstlisting}
>>> T = Bunch
>>> t = T(left=T(left="a", right="b"), right=T(left="c"))
>>> t.left
{'right': 'b', 'left': 'a'}
>>> t.left.right
'b'
>>> t['left']['right']
'b'
>>> "left" in t.right
True
>>> "right" in t.right
False
\end{lstlisting}

Конечно же этот шаблон можно использовать не только для создания деревьев. Он пригодится в любой ситуации, где необходим гибкий объект, умеющий задавать свои атрибуты при создании.
\end{notice}

\subsection{Множество разных представлений}

\newpage
\section{Beware of Black Boxes}
\subsection{Hidden Squares}
\subsection{The Trouble with Floats}
\section{Заключение}
\section{Если вы заинтересовались…}
\section{Упражнения}
\section{Ссылки}





%%% Local Variables: 
%%% mode: latex
%%% TeX-master: "mapl"
%%% End: 
 %% The Basics
%%% Local Variables: 
%%% mode: latex
%%% TeX-master: "mapl"
%%% End: 


\end{document}

%%% Local Variables: 
%%% mode: pdflatex
%%% TeX-master: t
%%% End: 
